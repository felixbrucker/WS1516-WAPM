\documentclass[twoside,12pt]{scrartcl}

%Encoding und Umlaute
\usepackage[utf8]{inputenc}
\usepackage[T1]{fontenc}
\usepackage{lmodern}

%Sprache
\usepackage[ngerman]{babel}

%Seitenränder
\usepackage{geometry}
\geometry{a4paper,left=4cm,right=25mm, top=25mm, bottom=25mm}

%Zeilenabstand
\usepackage{setspace}

%Matheumgebung
\usepackage{amsmath}

%Bilder
\usepackage{graphicx}

%Aufzählungen
\usepackage{enumitem}

%Literaturverzeichnis
\usepackage{url}
\usepackage[numbers,square,compress,merge,sort]{natbib}


%Textumflossene Bilder und Unterschriften
\usepackage{subcaption}

% Tabellen
\usepackage{multirow}
\usepackage[usestackEOL]{stackengine}
\usepackage{tabularx}
\usepackage[flushleft]{threeparttable}

%Farben
\usepackage[dvipsnames,svgnames]{xcolor}

%Kopf- und Fußzeilen
%\usepackage{fancyhdr}
%\pagestyle{fancy}
\usepackage{scrpage2}


%%Setzt Schrift auf helvetica
\usepackage{helvet}
\renewcommand{\familydefault}{\sfdefault}


%Quellcode
\usepackage{listings}

%%%%%%%%%%%%%%%%%%%% Quelltext Element - Style-File %%%%%%%%%%%%%%%%%%%%%%%
\usepackage[varqu]{zi4}
\definecolor{eclipse-red}{RGB}{127,0,85}
\definecolor{eclipse-blue}{RGB}{0,0,192} %for fields
\definecolor{eclipse-strings}{RGB}{42,0,255}
\definecolor{eclipse-green}{RGB}{63,127,95}
\definecolor{eclipse-lightblue}{RGB}{127,159,191} %for Javadoc Tags
\definecolor{eclipse-gray}{RGB}{100,100,100} %for annotations
\definecolor{eclipse-JavadocHTML}{RGB}{127,127,159} %for Javadoc HTML
\definecolor{eclipse-JavadocLinks}{RGB}{63,63,191}
\definecolor{eclipse-Javadoc}{RGB}{63,95,191}
\lstdefinestyle{eclipse-java}{%
	basicstyle=\ttfamily\small,%\fontfamily{zi4}\fontsize{10pt}{10pt}\selectfont,%
	tabsize=4,%
	breakautoindent=true,%
	breaklines,%
	postbreak=\space\space\space,%
	breakindent=0pt,%
	keywordstyle=\color{eclipse-red}\bfseries,%
	stringstyle=\color{eclipse-strings},%
	commentstyle=\color{eclipse-green},
	showstringspaces=false,
	morecomment=*[s][\color{eclipse-Javadoc}]{/**}{*/},
	morecomment=*[l][commentstyle]{//},
	language=Java,morekeywords={enum},
	emph={field,field2},emphstyle={\color{eclipse-blue}},
	emph=[2]{RED,GREEN,BLUE,staticField,inheritedField},emphstyle=[2]{\color{eclipse-blue}\itshape},
	emph=[3]{staticMethod},emphstyle=[3]\itshape,
	emph={[4]TASK},emphstyle={[4]\color{eclipse-lightblue}\bfseries},
	alsoletter={@},
	emph=[5]{@author,@deprecated},emphstyle=[5]{\color{eclipse-lightblue}\bfseries},
	moredelim=[l][\color{eclipse-gray}]{@},
	moredelim=*[s][\color{eclipse-Javadoc}]{/**}{*/},
	moredelim=[s][\color{eclipse-JavadocLinks}]{@link}{*},
	extendedchars=true,%
	literate={Ö}{{\"O}}1 {Ä}{{\"A}}1 {Ü}{{\"U}}1 {ß}{{\ss}}1 {ü}{{\"u}}1
	{ä}{{\"a}}1 {ö}{{\"o}}1}


%Glossar, Abkürzungsverzeichnis, Symbolverzeichnis
\usepackage[nonumberlist, acronym, toc]{glossaries}


%Ein eigenes Symbolverzeichnis erstellen
\newglossary[slg]{symbolslist}{syi}{syg}{List of symbols}

%Den Punkt am Ende jeder Beschreibung deaktivieren
\renewcommand*{\glspostdescription}{}

%Glossar-Befehle anschalten
\makeglossaries


%Diese Befehle sortieren die Einträge in den
%einzelnen Listen:
%makeindex -s datei.ist -t datei.alg -o datei.acr datei.acn
%makeindex -s datei.ist -t datei.glg -o datei.gls datei.glo
%makeindex -s datei.ist -t datei.slg -o datei.syi datei.syg


\newglossaryentry{netzteil}{name=Netzteil, description={Stromversorgung verschiedener Geräte}, plural=Netzteile}
\newglossaryentry{botnetz}{name=Botnetz, description={Infrastruktur zur Durchführung netzbasierter Angriffe}}


\newacronym{isp}{ISP}{Internet Service Provider}
\newacronym{cep}{CEP}{Complex Event Processing}


\newcommand*{\myglossaryindent}{0cm}

\newglossarystyle{longwithindent}{%
	\glossarystyle{long}%
	\renewenvironment{theglossary}%
	{\begin{longtable}[l]{@{\hspace{\myglossaryindent}}lp{\glsdescwidth}lp{\glspagelistwidth}@{}}}%
		{\end{longtable}}%
} 



\newglossaryentry{symb:Pi}{
	name=$\pi$,
	description={Die Kreiszahl.},
	sort=symbolpi, type=symbolslist
}
\newglossaryentry{symb:Phi}{
	name=$\varphi$,
	description={Ein beliebiger Winkel.},
	sort=symbolphi, type=symbolslist
}
\newglossaryentry{symb:Lambda}{
	name=$\lambda$,
	description={If marginal value is exceeded, the message is classified as spam.},
	sort=symbollambda, type=symbolslist
}


\usepackage{blindtext}
\pagenumbering{arabic}
\pagenumbering{Roman}


\title{Ein Testdokument}
\author{Jessica Steinberger}
\date{\today}


\begin{document}
	
	
%	\fancyhead{} % clear all header fields
%	\fancyhead[RO,LE]{\bfseries Mein erstes \LaTeX-Dokument}
%	\fancyhead[LO,RE]{\leftmark}
%	\fancyfoot{} % clear all footer fields
%	\fancyfoot[LE,RO]{\thepage}
%	\renewcommand{\headrulewidth}{0.4pt}
%	\renewcommand{\footrulewidth}{0.4pt}
	
	\pagestyle{scrheadings}
	\clearscrheadfoot
	\automark[chapter]{section}
	\ihead{\bfseries Mein erstes \LaTeX-Dokument}
	\ohead{\leftmark}
	\ifoot{\pagemark} 
	\setheadsepline{1pt} 
	\setfootsepline{1pt}
	
	
	\maketitle
	
	%Inhaltsverzeichnis
	\tableofcontents
	
	\newpage
	
	%Abkürzungsverzeichnis
	\printglossary[type=\acronymtype,style=longwithindent]
	\newpage
	
	%Symbolverzeichnis
	\printglossary[type=symbolslist,style=long]
	
	\newpage
	
	\section{Abstract}
	
	Große klassische Netzwerke sind sehr aufwendig zu konfigurieren und an neue Bedingungen (qos) anzupassen. Die Idee des Software-defined-Networks (SDN), des programmierbaren Netzwerkes, löst viele dieser Probleme. Durch die lose Kopplung von Hardware, die das eigentliche Weiterleiten der Datenpakete übernimmt, und Software, die zur Steuerung der Hardware dient, wird eine Abstraktion der Hardware, ähnlich zu Hypervisoren bei der Virtualisierung von Rechnern, geschaffen und eine einfach umsetzbare Programmierung des Netzwerkes ermöglicht. In dieser Arbeit wird eine strukturierte Übersicht über Software-defined-Networks gegeben. Wir stellen die historische Entwicklung von SDN dar. Des Weiteren geben wir eine Übersicht über die Architektur von SDN und zeigen Anwendungen für SDN auf. Schließlich wird noch ein Ausblick für die Zukunft gegeben.
	
	\section{Einleitung}
	

	% Zeilenabstand auf 1.5 setzten
	\onehalfspacing
	
	
	Computernetzwerke bestehen traditionell aus Switchen und Routern, die den Netzwerkverkehr weiterleiten, und sonstigen Geräten wie z.B. Firewalls, die lediglich Netzwerkverkehr verändern in dem z.B. bestimmter Verkehr geblockt wird oder bestimmte qos Regeln angewendet werden. Bei solchen Netzen ist der Konfigurationsaufwand der Administratoren sehr hoch, da viele Aufgaben manuell und womöglich noch pro Gerät durchgeführt werden müssen. Auch das „Übersetzen“ von komplexen Anforderungen in maschinenlesbare Befehle muss manuell gemacht werden.
Aus dem Wunsch zu vereinfachten Netzwerkadministrationsmöglichkeiten entstand das „programmierbare Netzwerk“ oder auch Software-Defined-Network (SDN). Bei Software-defined-Networks ist die Hardware, die den Netzwerkverkehr weiterleitet, von der Software, die die Entscheidungen zum Weiterleiten trifft, getrennt. Dadurch lassen sich allgemeine Schnittstellen für Entwickler definieren, die eine einfache Anpassung des Netzwerks erlauben.
Dieses Arbeit ist wie folgt strukturiert: Teil 2 gibt eine historische Übersicht über SDN und zeigt die Entwicklung auf, Teil 3 stellt die Architektur von SDN dar, Teil 4 beschreibt Anwendungen im Bezug zu SDN und Teil 5 gibt Aussichten für die Zukunft.
	\section{Historie / Entwicklung der SDN}
	
	Es gab einige Vorreiter des SDN die hier aufgelistet werden:
	
	\begin{itemize}
		\item General Switch Management Protocol (GSMP), ein von der IETF 1996 spezifiziertes Protokoll zur einfachen Verwaltung von Switchports, Anforderungskonfigurationen, Anforderungsstatistiken etc.
		\item Active Networking, die Idee von programmierbaren Switchen durch versenden von Programmiercode an den Switch.
		\item NETCONF, ein 2006 von der IETF vorgestelltes Protokoll zur Verwaltung von Netzwerkgeräten.
		\item Ethane, Vorgänger von OpenFlow, ein Modell bei dem ein zentraler Controller die Richtlinien und Sicherheit eines Netzwerks verwaltet.
	\end{itemize}
	
	\section{Architektur von SDN}
	\begin{itemize}
		\item Aktuelle SDN Architektur
		\begin{itemize}
			\item ForCES: Bei Forward and Control Element Separation (ForCES) wird die Control Plane in unmittelbarer Umgebung behandelt, aber nicht zwingend auf dem Gerät selbst. Diese wird auch Control Element (CE) genannt, das Element zum Weiterleiten der Daten wird als Forwarding Element (FE) bezeichnet. Der Logical Function Block (LFB) ist eine Kontrolleinheit im FE die vom CE über das ForCES Protokoll konfiguriert wird.
			\item OpenFlow: Im Gegensatz dazu wird bei OpenFlow die Control Plan ganz vom Gerät entfernt und zentral auf einem weiteren Computer ausgeführt. Die Kommunikation von Gerät und Controller läuft über das OpenFlow Protokoll. Das Gerät besitzt eine sogenannten Flow Table (FT) in der die folgenden Felder existieren: 
			\begin{itemize}
				\item „match fields“, enthält Informationen zum Finden von Paketen für diese Regel
				\item „counter“, enthält Statistiken über den jeweiligen Flow
				\item „set of instructions“, enthält die Befehle die auf die Pakete des Flows angewendet werden sollen
				\end{itemize}
Beim Eintreffen eines Paketes am Gerät wird zunächst in der FT nach einem Treffer gesucht, ähnlich wie bei einer Routingtabelle, und die entsprechenden Aktionen die im „set of instructions“ Feld der Tabelle stehen werden auf das Paket angewendet.
TODO Figur 1 zeigt den Ablauf eines einkommenden Pakets.
		\end{itemize}
		\item Controller
		\begin{itemize}
			\item Das größte Bedenken bei ausgelagerten Controllern ist, dass die Performance und Skalierbarkeit darunter leidet. Aktuelle Tests mit den OpenFlow Controllern zeigen, dass diese durchaus bei großen Netzwerken mit mindestens 50000 neuen Flow-Requests pro Sekunde zurechtkommen. Es gibt bereits hoch-skalierbare Controller (NOX-MT) [1] die acht Kerne nutzen und so 1,6 Millionen Flow-Requests pro Sekunde bei sehr geringer Latenz verarbeiten können. Es können weiterhin mehrere Conbtroller genutzt werden, um zum einen die Performance, als auch die Ausfallsicherheit, was bei einem zentralen System von großer Bedeutung ist, zu verbessern. Es gibt auch dezentralisierte Controller, wie zum Beispiel Onix und HyperFlow, dort werden logisch zentrale Controller aus physikalisch dezentralen Controllern erstellt. Ebenfalls gibt es Hybridlösungen, bei denen zum einen lokale Controller für lokale Entscheidungen, als auch globale zentrale Controller für weiterreichende Entscheidungen kontaktiert werden.
		\end{itemize}
		\item Southbound-API
		\begin{itemize}
			\item Die Kommunikation zwischen Gerät und Controller wird auch als „Southbound Communication“ bezeichnet. In OpenFlow erfolgt diese mit TLS verschlüsselt.
		\end{itemize}
		\item Northbound-API
		\begin{itemize}
			\item Die Kommunikation von Controller mit externen Anwendungen  wird auch als „Northbound Communication“ bezeichnet. Die Schnittstelle wird ebenfalls für Controller-Controller Kommunikation genutzt.
		\end{itemize}
		\item Standardarisierung
		\begin{itemize}
			\item Die IETF hat an Standardisierungen für ForCES im Bezug zu Protokollen sowie Mechanismen und Schnittstellen gearbeitet.
			\item Die Open Network Foundation (ONF) hat an der Standardarisierung von OpenFlow gearbeitet.
		\end{itemize}
	\end{itemize}

Fig 1

\section{SDN Anwendungen}
\begin{itemize}
	\item Enterprise Netzwerke
	Große Unternehmen haben sehr große Netzwerke die ein hohes Maß an Sicherheit verlangen. SDN können genutzt werden um Firewalls, Loadbalancer, NATs sowie Netzwerk ACLs zu ersetzen. Durch diese Umstellung fallen fast alle sogenannten „Middleboxes“ weg, was zu einer deutlichen Vereinfachung der Konfiguration sowie des Verwaltungsaufwandes führt.
	\item Rechenzentren
	Die Netzwerke in Rechenzentren sind oft überdimensioniert um einem hohen Peak an Anforderungen standzuhalten. Daraus resultiert, dass diese im Allgemeinen nicht alle Ressourcen ausnutzen. In solchen großen Rechenzentren spielt der Aspekt des Stromverbrauchs eine große Rolle, hier kann SDN helfen die Kosten zu reduzieren. Durch das Finden eines optimalen Netzwerkteils der für den aktuellen Bedarf ausreicht und das Ausschalten der nicht benötigten Geräte wird bis zu 60% an Energie gespart. Allerdings hat SDN auch Performanceeinbußen bei reinen high-performance Netzwerken, was durch das Ständige auslagern von Anfragen an den Controller zustande kommt. Im Gegenzug wurden weitere Frameworks vorgestellt, die einem solchen Verhalten, durch die Verlagerung der meisten Flows zurück in den Switch, entgegenwirken. Ein praktisches Beispiel für den Einsatz von OpenFlow verbundenen Rechenzentren wurde 2012 von Google beim Open Network Summit vorgestellt, hierbei liefen viele Links bei annährend 100% Auslastung.
	\item Heim- und Kleinbetriebe
	Gerade durch die niedrigen Kosten von Netzwerkgeräten haben viele kleine Firmen bereits ein relativ komplexes Netzwerk. Um nicht durch Malware verseucht zu werden spielt auch die Sicherheit eine große Rolle und eine Änderung der Konfiguration des Netzes sollte keine Ausfälle verursachen. Oft ist es auch nicht möglich in jedem Büro einen Netzwerkadministrator zu haben. Hier könnte ein SDN, durch remote verwalteten Switche und Verteiltes Monitoring zur Erkennung von Sicherheitsproblemen, helfen.
\end{itemize}

\section{Zukunft}
\begin{itemize}
	\item Controller und Switch
	SDN haben Skalierungs-, Performance- und Sicherheitsherausforderungen, welche bereits durch Forschungsarbeiten aufgearbeitet und minimiert werden. Physikalisch dezentrale aber logisch zentrale Controller erweisen sich als wesentlich robuster und besser skalierbar als ein zentraler Controller, Beispiele solcher Controller sind Onix, Kando und HyperFlow.
	\item Das Internet mit SDN
	Die Natur eines zentralen Controllers widerspricht der dezentralen Struktur des Internets, allerdings ist eine logisch zentrale, aber physikalisch dezentrale Struktur der beteiligten „Autonomous Systems“ (AS) denkbar. Es gibt bereits ein paar Versuche dieser Idee die auf „Multi protocol label switching“ (MPLS) basieren und Aufgaben an Geräte innerhalb des Bereichs oder außerhalb des Bereichs weitergeben. Ein weiterer Ansatz nutzt NOX und OpenFlow  um BGP ähnliche Funktionalität für das Routing innerhalb eines AS nutzbar zu machen.
	\item Virtualisierung
	Clouddienstleister und Virtualisierungsanbieter haben hohe Anforderungen an ihre Netzwerkinfrastruktur, schließlich müssen neue VMs oder Container innerhalb kürzester Zeit bereitgestellt werden. Ebenfalls muss ein hoch skalierbares Netzwerk-backend vorhanden sein, dass sich dynamisch den Anforderungen anpasst. Hierfür gibt es bereits Lösungen wie zum Bespiel Floodlight welches in OpenStack genutzt werden kann, oder NOX für MirageOS. LIME, ein Design zum effizienten migrieren von Virtuellen Maschinen wurde ebenfalls vorgestellt.
\end{itemize}

\section{Zusammenfassung}


\begin{figure}
	\centering
	\begin{subfigure}[t]{.3\linewidth}
		\includegraphics[width=.95\linewidth]{F1}
		\caption{Funk}
		\label{fig:sub1}
	\end{subfigure}%
	\hfill
	\begin{subfigure}[t]{.3\linewidth}
		\includegraphics[width=.95\linewidth]{F2}
		\caption{Start}
		\label{fig:sub2}
	\end{subfigure}
	\hfill
	\begin{subfigure}[t]{.3\linewidth}
		\includegraphics[width=.95\linewidth]{F3}
		\caption{Sonnenaufgang}
		\label{fig:sub3}
	\end{subfigure}%
	\caption{Paragliding an der Wasserkuppe}
	\label{fig:paragliding}
\end{figure}
	
	\vspace{1cm}
	
	Abbildung \ref{fig:sub1} zeigt, ...
	
	
	\vspace{1cm}
	
	
	
	
	
	
	
	
	
	
	
	
	
	
	
	
\vspace{0.5cm}

Eine Studie von \citet{Hofstede2014} zeigt, dass....

\begin{lstlisting}[language=Java, caption={Matrix},label=alg_euklid,style=eclipse-java,moreemph={[2]out}]
public class Array {
	
	public static void main(String[] args) {
		int [][] primes = {{2,3,5}, {7,11,13}};
		
		for (int i = 0; i < primes.length; i++) {
			for (int j = 0; j < primes[i].length; j++) {
				System.out.print(primes[i][j]+" ");
			}
			System.out.println();
		}
	}
}
\end{lstlisting}

\subsection{Abschnitt}

	\blindtext
	
	\vspace{1cm}
	Der Algorithmus in Listing \ref{alg_euklid} zeigt, ...
		\vspace{1cm}
	
	\begin{table}[!t]
		\centering
		\renewcommand{\arraystretch}{1.3}
			\caption{Overview of exchange protocols and formats \cite{Steinberger2015}}
			\label{table_exchangeprotocols}
			\begin{tabular}{|l|c|c|c|}
				\hline
				\textbf{Protocol} &  \textbf{OSI layer} & \textbf{Format} & \textbf{Security} \\
				\hline
				CIDF & Transport  & CISL messages  & \addstackgap{\shortstack{symmetric \\ cryptography}} \\
				\hline
				RID & Application & IODEF & TLS \\
				\hline
				XEP-0268 & Application & IODEF & TLS \\
				\hline
				IDXP & Application & IDMEF & TLS\\
				\hline
				CLT & Transport & CEE &  \addstackgap{\shortstack{provided by syslog \\(RFC $5425$)}}\\
				\hline
				\multirow{4}{*}{SMTP} & \multirow{4}{*}{Application} & \multirow{4}{*}{ \addstackgap{\shortstack{CAIF \\ ARF \\ x-arf}}} & None\\
				&  &  & S/MIME\\
				&  & & Multipart/Signed \\
				& & &  Multipart/Encrypted \\
				\hline
				syslog (RFC $3164$) & Transport & syslog (RFC $3164$) & none \\
				\hline
				syslog (RFC $5425$) & Transport & syslog (RFC $5424$) & TLS \\
				\hline
			\end{tabular}
	\end{table}	
	
		\blindtext
		\vspace{1cm}
	
	\begin{tabular*}{\textwidth}{@{\extracolsep{\fill}}|l|c|r|}
		\hline
		Links & Zentriert & Rechts \\ \hline
		L & Z &R  \\\hline
	\end{tabular*}
	
		\vspace{1cm}
	\blindtext
	\subsection{Abschnitt}
	
		\vspace{1cm}
		
		\begin{tabularx}{\textwidth}{@{\extracolsep{\fill}}|X|X|X|}
			\hline
			Links & Zentriert & Rechts \\ \hline
			L & Z &R  \\\hline
		\end{tabularx}
	\vspace{1cm}
	\blindtext
	
	\begin{table}[!t]
		\centering
		\resizebox{\textwidth}{!}{
		\renewcommand{\arraystretch}{1.3}
		\begin{threeparttable}
			\caption{Evaluation summary of the exchange formats}
			\label{table_summary}
			\centering
			\begin{tabular}{|l|c|c|c|c|c|c|c|c|c|c|}
				\hline
				\textbf{Criterion} &  \textbf{CIDF} & \textbf{IODEF} & \textbf{CAIF} & \textbf{IDMEF} &  \textbf{ARF} &  \textbf{CEE}&  \multicolumn{2}{c|}{\textbf{X-ARF}} & \multicolumn{2}{c|}{\textbf{Syslog}}\\
				&&&&&&& v$0.1$ & v$0.2$ & RFC $3164$ & RFC $5425$\\
				\hline
				Interoperability & $-$& $-$ & $-$ & $-$ & $+$ & $+$ & $+$ & $+$ & $+$ & $+$\\
				\hline
				Extensibility & $+$ & $+$ & $+$ & $+$ & $+$ & $+$ & $+$ & $+$ & $+$ & $+$\\
				\hline
			\end{tabular}
			\begin{tablenotes}
				\small
				\item Legend: high ($+$), medium ($0$) and low ($-$)
			\end{tablenotes}
		\end{threeparttable}}
	\end{table}
	
	\vspace{1cm}
	
	Abbildung \ref{fig:Grundkurs_Suedhang} zeigt ...
	
	Tabelle \ref{table_summary} zeigt ...
	
	\section{Kapitel}
	
	Lorem ipsum dolor $x^2$ sit amet, consectetuer \(x^2\) adipiscing elit. Aenean commodo ligula eget dolor. Aenean massa. Cum sociis natoque penatibus et magnis dis parturient montes, nascetur ridiculus mus. Donec quam felis, ultricies nec, pellentesque eu, pretium quis, sem. Nulla consequat massa quis enim. Donec pede justo, fringilla vel, aliquet nec, vulputate eget, arcu. In enim justo, rhoncus ut, imperdiet a, venenatis vitae, justo. Nullam dictum felis eu pede mollis pretium. Integer tincidunt. Cras dapibus. Vivamus elementum semper nisi. 
	
	\begin{align}
	x^2
	\end{align}	
	
	Aenean vulputate eleifend tellus. Aenean leo ligula, porttitor eu, consequat vitae, eleifend ac, enim. Aliquam lorem ante, dapibus in, viverra quis, feugiat a, tellus. Phasellus viverra nulla ut metus varius laoreet. Quisque rutrum. Aenean imperdiet. Etiam ultricies nisi vel augue. Curabitur ullamcorper ultricies nisi. Nam eget dui. Etiam rhoncus. 
	
	\begin{align*}
	x^2
	\end{align*}	
	
	Maecenas tempus, tellus eget condimentum rhoncus, sem quam semper libero, sit amet adipiscing sem neque sed ipsum. Nam quam nunc, blandit vel, luctus pulvinar, hendrerit id, lorem. Maecenas nec odio et ante tincidunt tempus. Donec vitae sapien ut libero venenatis faucibus. Nullam quis ante. Etiam sit amet orci eget eros faucibus tincidunt. Duis leo. Sed fringilla mauris sit amet nibh. Donec sodales sagittis magna. Sed consequat, leo eget bibendum sodales, augue velit cursus nunc, 
	
	
	\Blindtext
	
	\listoffigures
	
	\blindtext
	\subsection{Abschnitt}
	
	%Tabellenverzeichnis
	\listoftables
	
	\blindtext
	
	\section{Kapitel}
	\Blindtext
	
	%Quellcodeverzeichnis
	\lstlistoflistings
	
	%Glossar
	\printglossary[style=altlist,title=Glossar]
	\newpage
	
	%Literaturverzeichnis & Style
	\bibliographystyle{plainnat}			
	\bibliography{literature} 
	
\end{document}

